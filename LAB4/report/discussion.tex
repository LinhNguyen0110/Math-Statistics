\section{Обсуждение}

\subsection{Варьирование неопределенности изменений}

Почти все компоненты вектора $w$ оказались равны 1, то есть, расширения интервалов измерений почти не понадобилось. \\
Недостатком полученного значения с единичными значениями $w_{1}^i$ является неучет расстонияй точек регрессионной зависимости до данных интервальной выборки. Данная модель "не чувствует" отклонений измерений от прямой на концах выборки -- неопределенности измерений достаточно велики, чтобы покрыть этот эффект.

\subsection{Варьирование неопределенности изменений с расширением и сужением интервалов}

В результате построения модели небольшая часть интервалов (от 197 до 200) расширилась, остальные интервалы сузились. Сумма компонент вектора $w$ уменьшилась более чем в 6 раз. Таким образом, постановка задачи с возможностью одновременного расширения и сужения радиусов неопределенности измерений позволяет более гибко подходить к задаче оптимизации. \\
Из графиков векторов $w_0$ и $w_1$ можно сделать вывод, что график вектора $w_0$ содержит большее количество информации, чем график вектора $w_1$.\\

\subsection{Анализ регрессионных остатков}

Интервальные выборки остатков получились с весьма разными свойствами. Диаграмма рассеяния регрессионных остатков для модели с сужением и расширением интервалов выглядит более естественно, хотя из-за высокой степени совместности исходной выборки разница незначительна. \\
Данный вывод подтверждает и график частот элементарных подинтервалов регрессионных остатков при вычислении моды моделей. Графики очень похожи, однако график для частот элементарных подинтервалов регрессионных остатков при вычислении моды модели с расширением и сужением интервалов имеет немного более широкую внутреннюю оценку, что соответствует большей устойчивости к возмущениям данных.\\
Обе интервальные выборки остатков имеют довольно высокий коэффициент Жакара, что свидетельствует о высокой степени совместности выборки. 

\subsection{Информационное множество задачи} 

Информационное множество для представляет представляет собой многогранный брус, а интервальные оценки, именуемые интервальной оболочкой, задают брус, описанный вокруг многогранного множества.

\subsection{Коридор совместных зависимостей}

Из внешнего вида коридора совместных зависимостей можно сделать вывод, что внутри его можно провести множество прямых и что он покрывает почти всю интервальную выбору. 

\subsection{Прогноз вне области данных}

Была расширена область определения аргумента для модели и построен прогноз за пределами интервальной выборки. На основе полученных результатов можно сделать вывод, что величина неопределенности растет по мере удаления от области, в которой производились исходные измерения. Это обусловлено видом коридора зависимостей, расширяющимся за пределами области измерений.