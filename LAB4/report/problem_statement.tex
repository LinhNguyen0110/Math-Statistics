\section{Постановка задачи}

Имеется выборка данных с интервальной неопределенностью. Число отсчетов в выборке равно 200. Используется модель данных с  уравновешенным интервалом погрешности. \\

$\bm{x} = \stackrel{\circ}{x} + \bm{\epsilon}$; \quad $\bm{\epsilon} = [-\epsilon, \epsilon]$  для некоторого $\epsilon >0 $, \\

Здесь $\stackrel{\circ}{x}$ -- данные некоторого прибора, $\epsilon = 10 ^ {-4}$ -- погрешность прибора.

Нужно иллюстрировать данные выборки, построить диаграмму рассеяния, построить линейную регрессионную зависимость варьированием неопределенности изменений (без сужения и с расширением и сужением интервалов),  произвести анализ регрессионных остатков, построить информационное множество по модели, проиллюстрирвоать коридор совместных зависимостей, построить прогноз вне области данных. 

Файл с данными интервальной выборки "channel\_2\_900nm\_0\_23mm.csv" \quad расположен по следующей ссылке: \href{https://github.com/anivse/MathematicalStatistics/tree/main/4/source}{\textbf{исходные данные}} \\
Данные взяты из архива, расположенного по следующей ссылке: \href{https://github.com/AlexanderBazhenov/Solar-Data/blob/main/%D0%A1%D1%82%D0%B0%D1%82%D0%B8%D1%81%D1%82%D0%B8%D0%BA%D0%B0%20%D0%B8%D0%B7%D0%BC%D0%B5%D1%80%D0%B5%D0%BD%D0%B8%D0%B9.rar}{\textbf{архив с данными интервальных выборок}} \\(название использованного файла "Канал 2\_900nm\_0.23mm.csv")

